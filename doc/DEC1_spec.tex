\documentclass{article}

\usepackage{amsmath}


\newcommand{\escmathcom}[3]{ \newcommand{#1}[#2]{\mbox{#3}}}

\escmathcom{\mcal}{1}{$\mathcal{#1}$}
\escmathcom{\msf}{1}{$\mathsf{#1}$}
\escmathcom{\SEP}{0}{$ \ \triangleright \ $}
\escmathcom{\LANG}{0}{$ \langle \ $}
\escmathcom{\RANG}{0}{$ \ \rangle $}


\begin{document}

\title{DEC 1.0 specification}

\author{Paolo Torrini, David Nowak \\
  \small 2XS Team, CRIStAL-Universit{\'e} Lille 1 \\
   \texttt{https://www.cristal.univ-lille.fr} }

\date{\vspace{-2ex}}

\maketitle  

\section{Introduction}

DEC is an imperative functional language with bounded recursion and
generic effects deeply embedded in Coq. It has been designed as a
small intermediate language to support translation from monadic
Gallina code to C. It captures the expressiveness of the monadic code
used in the implementation of the Pip protokernel
(\texttt{https://github.com/2xs/pipcore}). Being defined in terms of
abstract datatypes, it allows for syntactic manipulation to be
implemented and verified directly in Coq. In particular, this is the
case for its translations to Coq representations of C, such as
CompCert C (\texttt{http://compcert.inria.fr}).

The DEC specification includes a definition of the syntax, definitions
of the static semantics, and a definition of the dynamic semantics,
which is stateful and based on a small-step transition relation, in
the style of SOS. Syntactic categories and semantic relations are
mainly defined in terms of inductive datatypes.

Essentially, DEC provides the expressiveness of a fragment of C, when
we restrict to C expressions and call by value. It allows for
sequencing (corresponding to C comma expressions), local variables,
conditional branching, internal and external function calls. As in C,
functions cannot be passed as arguments by value. Passing by reference
is not modelled. Internal functions are essentially primitive
recursive ones, defined by a bounded \emph{iterate}-style
construct. Any Gallina function can be imported as an external one,
and semantically treated as a generic effect with respect to the DEC
state.

Relying on Coq modules, the definition of DEC is parametric in the
type of the mutable state and in the type of identifiers. These
parameters are specified in the module type \emph{ModTyp.v}. With
respect to Pip, the concrete module \emph{BaseMod.v} provides an
instantiation of these parameters consistent with the current output
of the Digger tool (\texttt{https://github.com/2xs/digger}).

DEC types include primitive value types, which are specified by an
admissibility type class over Gallina types, and function types based
on primitive types. DEC programs do not bear type annotation, save for
formal parameters in functions.

DEC normal forms include data values and functions. DEC distinguishes
between data variables and function variables, while relying on a
single name space for identifiers. Correspondingly, DEC semantics
distinguishes between data environments and function environments, and
similarly for typing environments (or contexts). Environments are
generally implemented as association lists, functionally interpreted
as partial maps.

DEC values are defined by lifting Gallina terms and further hiding
their Gallina types, using dependent product. We call quasi-values the
head normal forms based on data, which simply include values and data
variables. Analogously, we call quasi-functions the head normal forms
based on functions, including functions and function variables.

Internal function definition includes local function variables, formal
parameters with their types, function bodies for zero and non-zero
fuel, an identifier as recursive call name (which can also be used as
function name), and a natural value that represents fuel (i.e. the
bound). DEC functions are structurally closed terms, in the sense that
each occurrence of an identifier in their bodies is bound either as a
formal parameter or as a local variable. In internal function calls,
concrete parameters are given as a lifted list of expressions.

The lifting constructors in the definition of expressions are
\msf{Val}, which lifts values, and \msf{Return} which lifts
quasi-values and is additionally tagged for external, non-semantic
reasons. \msf{bindN} is the sequencing constructor. The constructors
for local variables are \msf{bindS}, which allows for let-style
binding of identifiers to expressions, and \msf{bindMS}, which allows
for multiple binding of identifiers to normal forms. \msf{ifThenElse}
is the conditional branching constructor. The constructors for
internal and external function calls are \msf{apply} and \msf{modify},
respectively. Moreover, \msf{Var} and \msf{FVar} are used to lift
identifiers to the corresponding head-normal forms, and similarly
\msf{QV} and \msf{QF} to lift normal forms. \msf{FC} is the recursive
function constructor. The constructor \msf{PS} lifts expression lists
to parameters.

In the intended use of DEC, function definitions are known statically
from the function environment. That is, the \msf{bindMS} constructor
and the \msf{QF} lifting of functions to quasi-functions are only
meant to be used in the processing of programs. Moreover, the first
argument of \msf{ifThenElse} is meant to be a lifted quasi-value.

Notice that functions, quasi-functions, expressions, and parameters
are defined by mutual induction. The typing relation for each
syntactic category is also defined inductively, modulo the
well-typedness of the environments each object depends on. Usually,
the definition of a typing relation is parametric in the associated
typing contexts. Granted the intended use of DEC, we find it
convenient to define a more expressive, algorithmic typing relation,
which depends also on the function environment. In this way, the
definition of the typing relation for functions can be defined
inductively on the fuel.  This makes it possible to obtain a stronger
inductive principle for the mutually inductive datatype. Under the
restrictions we mentioned, the algorithmic typing relation we have
implemented is equivalent to the more usual, declarative one.

The dynamic semantics is specified in the style of structural
operational semantics (SOS) using a small-step transition relation on
configurations that for each category include a program term in that
category and a state of the mutable store. In general, the transition
relation depends on value environments and function environments. The
order of evaluation is fixed to left-to-right.



\section{Syntax}


\subsection{State type}

\begin{gather}
  \msf{W} \ : \ \msf{Type}
\end{gather}


\subsection{Identifier type}

\begin{gather}
  \msf{Id} \ : \ \msf{Type}
\end{gather}


\subsection{External function record}

\begin{gather}
\begin{array}{lll}  
  \mbox{\it xf} \ : & (\ \msf{XFun} \ (\mbox{\it InpT} \ : \ \msf{Type}) \
           (\mbox{\it OutT} \ : \ \msf{Type}) \ : \
  \msf{Type} \ := \\
  & \quad \{ \msf{b\_mod} \ : \ \msf{W} \to \mbox{\it InpT} \to (\msf{W} \ * \ \mbox{\it OutT}) \} \ )
\end{array}  
\end{gather}



\subsection{Data value types}

\begin{gather}
t \ : \ \msf{VTyp} \ \ := \ \msf{vtyp} \ \msf{Type}
\end{gather}




\subsection{Parameter types}


\begin{gather}
  ts \ : \ \msf{list} \ \msf{VTyp}
\end{gather}


\begin{gather}
  pt \ : \ ( \ \msf{PTyp} \ := \ \msf{PT} \ (\msf{list} \ \msf{VTyp}) \ )
\end{gather}


\subsection{Identifiers}

\begin{gather}
  x \ : \ \msf{Id} 
\end{gather}


\subsection{Data typing contexts}

\begin{gather}
  \begin{array}{lll}
\Gamma \ : \ ( \ \msf{ValTC} \quad := \quad \msf{list} \ (\msf{Id} \ *
\ \msf{VTyp}) \ ) \quad <: \quad  \msf{Id} \to \msf{option} \ \msf{VTyp}
\end{array}
\end{gather}  



\subsection{Function types}


\begin{gather}
(\tau \ : \ \msf{FTyp}) \ := \ \msf{ValTC} \Rightarrow \msf{VTyp}
\end{gather}



\subsection{Function typing contexts}

\begin{gather}
  \begin{array}{lll}
    \Phi \ : \ ( \ \msf{FunTC} \quad := \quad \msf{list} \ (\msf{Id} \ *
\ \msf{FTyp}) \ ) \quad <: \quad  \msf{Id} \to \msf{option} \ \msf{FTyp}
\end{array}
\end{gather}  



\subsection{Data values}

\begin{gather}
  (v \ : \ \msf{Value}) \ := \ \msf{cst} \ (T : \msf{Type}) \ T 
\end{gather}



\subsection{Quasi-values}

\begin{gather}
  (qv \ : \ \msf{QValue}) \ := \ \msf{Var} \ \msf{Id} \ \mid \ 
                               \msf{QV} \ \msf{Value} 
\end{gather}




\subsection{Data value environments}

\begin{gather}
\begin{array}{lll}  
  \rho \ : \ ( \ \msf{ValEnv} \quad := \quad \msf{list} \ (\msf{Id} \ *
  \ \msf{Value}) \ ) \quad <: \quad  \msf{Id} \to \msf{option} \ \msf{Value}
\end{array}
\end{gather}  




\subsection{Functions}

\begin{gather}
\begin{array}{lll}  
  (f \ : \ \msf{Fun}) \ := & \msf{FC} \
  (\mbox{\it local\_function\_definitions} \ : \ \msf{FunEnv}) \\
  & \quad (\mbox{\it formal\_parameters} \ : \ \msf{ValTC}) \\
  & \quad (\mbox{\it function\_body\_zero\_case} \ : \ \msf{Exp}) \\
  & \quad (\mbox{\it function\_body\_succ\_case} \ : \ \msf{Exp}) \\
  & \quad (\mbox{\it recursive\_call\_name} \ : \ \msf{Id}) \\
  & \quad (\mbox{\it recursion\_bound} \ : \ \msf{nat}
\end{array}  
\end{gather}


\subsection{Quasi-functions}

\begin{gather}
  (\mbox{\it qf} \ : \ \msf{QFun}) \ := \
  \msf{FVar} \ \msf{Id} \ \mid \ \msf{QF}
  \ \msf{Fun}
\end{gather}


\subsection{Function environments}

\begin{gather}
\begin{array}{lll}  
  \phi \ : \ ( \ \msf{FunEnv} \ := \ ( \msf{Id} \ * \ \msf{Fun}) \ )
   \quad <: \quad  \msf{Id} \to \msf{option} \ \msf{Fun}
\end{array}
\end{gather}  


\subsection{Return tags}

\begin{gather}
  (\mbox{\it tg} \ : \ \msf{Tag}) \ := \ \msf{LL} \ \mid \ \msf{RR} 
\end{gather}



\subsection{Expressions}

\begin{gather}
\begin{array}{lll}
  (e \ : \ \msf{Exp}) & := & 
%
   \msf{val} \ \msf{Value} \\
%
 & \mid & \msf{return} \ \msf{Tag} \ \msf{QValue} \\
%
  & \mid & \msf{bindN} \ \msf{Exp} \ \msf{Exp} \\
%
  & \mid & \msf{bindS} \ \msf{Id} \ \msf{Exp} \ \msf{Exp} \\
% 
  & \mid & \msf{bindMS} \ \msf{FunEnv} \ \msf{ValEnv} \ \msf{Exp} \\
%
  & \mid & \msf{ifThenElse} \ \msf{Exp} \ \msf{Exp} \ \msf{Exp} \\
%
  & \mid & \msf{apply} \ \msf{QFun} \ \msf{Prms} \\ 
%    
  & \mid & \msf{modify} \ (\msf{XFun} \ \msf{Type} \ \msf{Type}) \
   \msf{QValue}
\end{array}
\end{gather}



\subsection{Parameters}


\begin{gather}
  \mbox{\it vs} \ : \ \msf{list} \ \msf{Value}
\end{gather}


\begin{gather}
  \mbox{\it es} \ : \ \msf{list} \ \msf{Exp}
\end{gather}


\begin{gather}
  (ps \ : \ \msf{Prms}) \ := \ \msf{PS} \ (\msf{list} \ \msf{Exp})
\end{gather}




\subsection{Programs}

\begin{gather}
\begin{array}{lll}
  (\pi \ : \ \msf{Prog}) \ := \
%
  \msf{prog} \ \msf{Exp} 
%
   \mid \msf{define} \ \msf{Id} \ \msf{Fun} \ \msf{Prog} 
%
\end{array}
\end{gather}



\subsection{Configurations}

\begin{gather}
   (C \ : \ \msf{SynCat}) \ := \ \msf{QValue} \mid \msf{QFun} \mid
  \msf{Prms} \mid \msf{Exp} \mid \msf{Prog}
\end{gather}  



\begin{gather}
\begin{array}{llll}
    \msf{Config} \ C \ := \
    \LANG & \msf{state} : \msf{W} \ & \triangleright &
 \msf{program\_term} : C \quad \RANG
\end{array}
\end{gather}  



\subsection{Static sequents}

\begin{gather}
  \begin{array}{lll}
 \msf{Typing\_sequent} & := 
%
 \quad \msf{ValTC} \ \vdash \ \msf{QValue} \ :: \ \msf{VTyp} \\
%
& \mid \quad \ \vdash \ \msf{Value} \ :: \ \msf{VTyp} \\
%
& \mid \quad \ \vdash \ \msf{Fun} \ :: \ \msf{FTyp} \\
%
& \mid \quad \ \vdash \ \msf{option} \ \msf{Value} \ :: \ \msf{option}
 \ \msf{VTyp} \\
%
& \mid \quad \ \vdash \ \msf{option} \ \msf{Fun} \ :: \ \msf{option}
 \ \msf{FTyp} \\
%
& \mid \quad \ \vdash \ \msf{ValEnv} \ :: \ \msf{ValTC} \\
 %
& \mid \quad \ \vdash \ \msf{FunEnv} \ :: \ \msf{FunTC} 
\end{array}
\end{gather}  


\begin{gather}
  \begin{array}{lll}
 \msf{AlgTyping\_sequent} & := 
%
\ \msf{FunTC}; \msf{FunEnv} \ \vdash \ \msf{Prog} \ :: \ \msf{VTyp} \\
%
& \mid \quad \msf{FunTC}; \msf{FunEnv}; \msf{ValTC} \ \vdash
\ \msf{Exp} \ :: \ \msf{VTyp} \\
%
 & \mid \quad \msf{FunTC}; \msf{FunEnv}; \msf{ValTC} \ \vdash
\ \msf{Prms} \ :: \ \msf{PTyp} \\
%
& \mid \quad \msf{FunTC}; \msf{FunEnv} \ \vdash \ \msf{QFun} \ ::
\ \msf{FTyp}
\end{array}
\end{gather}  


\begin{gather}
  \begin{array}{lll}
 \msf{DecTyping\_sequent} & := 
%
\ \msf{FunTC} \ \vdash \ \msf{Prog} \ :: \ \msf{VTyp} \\
%
& \mid \quad \msf{FunTC}; \msf{ValTC} \ \vdash \ \msf{Exp} \ ::
\ \msf{VTyp} \\
%
 & \mid \quad \msf{FunTC}; \msf{ValTC} \ \vdash \ \msf{Prms} \ ::
 \ \msf{PTyp} \\
%
& \mid \quad \msf{FunTC} \ \vdash \ \msf{QFun} \ :: \ \msf{FTyp} 
\end{array}
\end{gather}  



\subsection{Dynamic sequents}


\begin{gather}
 \begin{array}{lll}
   \msf{Step\_sequent} & := \
   \msf{FunEnv} \ \vdash \ \msf{Config} \ \msf{Prog}
 \longrightarrow \msf{Config} \ \msf{Prog} \\
 %
 & \quad \quad \msf{FunEnv}; \msf{ValEnv} \ \vdash \ \msf{Config} \ \msf{Exp}
 \longrightarrow \msf{Config} \ \msf{Exp} \\
 %
 & \quad \quad \msf{FunEnv}; \msf{ValEnv} \ \vdash \ \msf{Config} \ \msf{Prms}
 \longrightarrow \msf{Config} \ \msf{Prms} \\
 %
 & \quad \quad \msf{ValEnv} \ \vdash \ \msf{Config} \ \msf{QValue}
 \longrightarrow \msf{Config} \ \msf{QValue} \\
%
 & \quad \quad \msf{FunEnv} \ \vdash \ \msf{Config} \ \msf{QFun}
 \longrightarrow \msf{Config} \ \msf{QFun}
\end{array}
\end{gather}  





\subsection{Relating value lists and lifted value lists}

\begin{gather}
\vdash \msf{values} \ : \ \msf{list} \ \msf{Exp} \to \msf{list}
\ \msf{Value} \to \msf{Prop}
\end{gather}

\begin{gather}
\frac{ \vdash \mbox{\it es} = \msf{map} \ \msf{val} \ \mbox{\it vs} }  
{\vdash \msf{values} \ \mbox{\it es} \ \mbox{\it vs}}
\end{gather}


\subsection{Typing context conversion to type lists}

\begin{gather}
\vdash \msf{tc2pt} \ : \ \msf{ValTC} \to \msf{PTyp} 
\end{gather}



\subsection{Value environment creation}

\begin{gather}
\vdash \msf{mkValEnv} \ : \ \msf{ValTC} \to \msf{list} \ \msf{Value} \ \to
 \msf{ValEnv}
\end{gather}



\section{Static semantics}


\subsection{Data values}

\begin{gather}
\frac{ \vdash T \ : \ \msf{Type} \ \qquad \ \vdash n \ : \ T}
 { \vdash \msf{cst} \ T \ n \ :: \ \msf{vtyp} \ T }
\end{gather}



\subsection{Option-lifted normal forms}

\begin{gather}
\frac{ }
 { \vdash \msf{None} \ :: \ \msf{None} }
\end{gather}


\begin{gather}
\frac{ \vdash v \ :: \ t}
 { \vdash \msf{Some} \ v \ :: \ \msf{Some} \ t }
\end{gather}


\begin{gather}
\frac{ \vdash f \ :: \ \tau}
 { \vdash \msf{Some} \ f \ :: \ \msf{Some} \ \tau }
\end{gather}




\subsection{Environments}


\begin{gather}
  \frac{ \bigwedge ( x: \msf{Id} ) \ ( \ \vdash \rho \ x \ :: \ \Gamma
    \ x \ ) } { \vdash \rho \ :: \ \Gamma}
\end{gather}

\begin{gather}
  \frac{ \bigwedge ( x: \msf{Id} ) \ ( \ \vdash \phi \ x \ :: \ \Phi
    \ x \ ) } { \vdash \phi \ :: \ \Phi }
\end{gather}





\subsection{Quasi-values}

\begin{gather}
\frac{ \vdash \Gamma \ x \ = \ \msf{Some} \ t}{ \Gamma \vdash
  \msf{Var} \ x \ :: \ t}
\end{gather}

\begin{gather}
\frac{ \vdash v \ :: \ t}{ \Gamma \vdash \msf{QV} \ v \ :: \ t}
\end{gather}





\section{Algorithmic static semantics}



\subsection{Expression lists}

\begin{gather}
\frac{ } { \Phi; \phi; \Gamma \vdash \msf{nil} \ :: \ \msf{nil} }
\end{gather}


\begin{gather}
\frac{ \Phi; \phi; \Gamma \vdash e \ :: \ t \qquad \Phi; \phi; \Gamma
  \vdash \mbox{\it es} \ :: \ ts } { \Phi; \phi; \Gamma \vdash
  \msf{cons} \ e \ \mbox{\it es} \ :: \ \msf{cons} \ t \ ts }
\end{gather}




\subsection{Parameters}


\begin{gather}
  \frac{\begin{array}{lll}
    \Phi; \phi; \Gamma \vdash \mbox{\it es} \ :: \ ts 
    \end{array}
  }{ \Phi; \phi; \Gamma \vdash \msf{PS} \ \mbox{\it es} \ ::
    \ \msf{PT} \ ts }
\end{gather}



\subsection{Functions}


\begin{gather}
  \frac{
\begin{array}{lll}  
  \vdash \phi \ :: \ \Phi  \\
  \Phi; \phi; \Gamma \vdash e_0 \ :: \ t \\ 
\end{array}  
  }{ \vdash \msf{FC} \ \phi \ \Gamma \ e_0 \ e_1 \ x \ 0 \ :: \ \Gamma
    \Rightarrow t}
\end{gather}

\begin{gather}
  \frac{
\begin{array}{lll}  
  \vdash \phi \ :: \ \Phi \qquad \vdash n \ : \ \msf{nat} \\ \vdash
  \msf{FC} \ \phi \ \Gamma \ e_0 \ e_1 \ x \ n \ :: \ \Gamma
  \Rightarrow t \\ (x, \ \Gamma \Rightarrow t) / \Phi; (x, \ \msf{FC}
  \ \phi \ \Gamma \ e_0 \ e_1 \ x \ n) / \phi; \Gamma \vdash e_1 \ ::
  \ t \\
\end{array}  
  }{ \vdash \msf{FC} \ \phi \ \Gamma \ e_0 \ e_1 \ x \ (\msf{S} \ n)
    \ :: \ \Gamma \Rightarrow t}
\end{gather}




\subsection{Quasi-functions}


\begin{gather}
\frac{\vdash \phi \ :: \ \Phi \qquad \vdash \Phi \ x \ = \ \tau}{\Phi;
  \phi \vdash \msf{FVar} \ x \ :: \ \tau}
\end{gather}

\begin{gather}
\frac{ \vdash f \ :: \ \tau}{ \Phi; \phi \vdash \msf{QF} \ f \ :: \ \tau}
\end{gather}



\subsection{Expressions}

\begin{gather}
\frac{ \vdash v \ :: \ t}{\Phi; \phi; \Gamma \vdash \msf{val} \ v ::
  \ t}
\end{gather}

\begin{gather}
  \frac{ \Gamma \vdash qv \ :: \ t \qquad \vdash \mbox{\it tg} \ :
    \ \msf{Tag} } {\Phi; \phi; \Gamma \vdash \msf{return} \ \mbox{\it
      tg} \ qv \ :: \ t}
\end{gather}

\begin{gather}
 \frac{\Phi; \phi; \Gamma \vdash e_1 \ :: \ t_1 \qquad \Phi; \phi;
   \Gamma \vdash e_2 \ :: \ t_2} {\Phi; \phi; \Gamma \vdash
   \msf{bindN} \ e_1 \ e_2 \ :: \ t_2 }
\end{gather}

\begin{gather}
 \frac{\Phi; \phi; \Gamma \vdash e_1 \ :: \ t_1 \qquad \Phi; \phi; (x,
   \ t_1) / \Gamma \vdash e_2 \ :: \ t_2} {\Phi; \phi; \Gamma \vdash
   \msf{bindS} \ x \ e_1 \ e_2 \ :: \ t_2 }
\end{gather}

\begin{gather}
  \frac{ \vdash \phi' \ :: \ \Phi' \qquad \vdash \rho \ :: \ \Gamma'
    \qquad \Phi' / \Phi; \phi' / \phi; \Gamma' / \Gamma \vdash e \ ::
    \ t} {\Phi; \phi; \Gamma \vdash \msf{bindMS} \ \phi' \ \rho \ e
    \ :: \ t }
\end{gather}

\begin{gather}
  \frac{\Phi; \phi; \Gamma \vdash e_1 \ :: \ \msf{vtyp} \ \msf{bool}
    \qquad \Phi; \phi; \Gamma \vdash e_2 \ :: \ t \qquad \Phi; \phi;
    \Gamma \vdash e_3 \ :: \ t } {\Phi; \phi; \Gamma \vdash
    \msf{ifThenElse} \ e_1 \ e_2 \ e_3 \ :: \ t }
\end{gather}

\begin{gather}
\frac{ \Phi; \phi \vdash \msf{\it qf} \ :: \ \Gamma' \Rightarrow t
  \qquad \Phi; \phi; \Gamma \vdash ps \ :: \ \msf{tc2pt}
  \ \Gamma'}{\Phi; \phi; \Gamma \vdash \ \msf{apply} \ \msf{\it qf}
  \ ps \ :: \ t}
\end{gather}

\begin{gather}
  \frac{ \vdash \mbox{\it xf} \ : \ \msf{XFun} \ T_1 \ T_2 \ \qquad
    \ \Gamma \vdash qv \ :: \ \msf{vtyp} \ T_1 } {\Phi; \phi; \Gamma
    \vdash \msf{modify} \ \mbox{\it xf} \ qv \ :: \ \msf{vtyp} \ T_2}
\end{gather}


\subsection{Programs}

\begin{gather}
 \frac{\vdash \phi \ :: \ \Phi \qquad \Phi; \phi \vdash e \ :: \ t}
      {\Phi; \phi \vdash \msf{prog} \ e \ :: \ t }
\end{gather}


\begin{gather}
 \frac{\vdash \phi \ :: \ \Phi \qquad \vdash f \ :: \ \tau \qquad (x,
   \ \tau) / \Phi; (x, \ f)/\phi \vdash \pi \ :: \ t} {\Phi; \phi
   \vdash \msf{define} \ x \ f \ \pi \ :: \ t }
\end{gather}




\section{Dynamic semantics}

\subsection{Var}

\begin{gather}
  \frac{ \vdash \rho \ x \ = \ \msf{Some} \ v}{ \rho \vdash \LANG w
    \SEP \msf{Var} \ x \RANG \longrightarrow \LANG w \SEP \msf{QV} \ v
    \RANG }
\end{gather}



\subsection{FVar}


\begin{gather}
  \frac{ \vdash \phi \ x \ = \ \msf{Some} \ f}{ \phi \vdash \LANG w
    \SEP \msf{FVar} \ x \RANG \longrightarrow \LANG w \SEP \msf{QF}
    \ f \RANG }
\end{gather}


\subsection{return}

\begin{gather}
  \frac{}{ \phi; \rho \vdash \LANG w \SEP \msf{return} \ \mbox{\it tg}
    \ v \RANG \longrightarrow \LANG w \SEP \msf{val} \ \mbox{\it tg}
    \ v \RANG }
\end{gather}


\begin{gather}
  \frac{ \rho \vdash \LANG w \SEP q \RANG \longrightarrow \LANG w'
    \SEP q' \RANG } { \phi; \rho \vdash \LANG w \SEP \msf{return}
    \ \mbox{\it tg} \ q \RANG \longrightarrow \LANG w' \SEP
    \msf{return} \ \mbox{\it tg} \ q' \RANG }
\end{gather}




\subsection{bindN}

\begin{gather}
  \frac{}{ \phi; \rho \vdash \LANG w \SEP \msf{bindN} \ (\msf{val}
    \ v) \ e \RANG \longrightarrow \LANG w \SEP e \RANG }
\end{gather}


\begin{gather}
  \frac{ \phi; \rho \vdash \LANG w \SEP e_1 \RANG \longrightarrow
    \LANG w' \SEP e'_1 \RANG }{ \phi; \rho \vdash \LANG w \SEP
    \msf{bindN} \ e_1 \ e_2 \RANG \longrightarrow \LANG w' \SEP
    \msf{bindN} \ e'_1 \ e_2 \RANG }
\end{gather}




\subsection{bindS}

\begin{gather}
  \frac{ }{
\begin{array}{lll}
    \phi; \rho \vdash \LANG w \SEP \msf{bindS} \ x \ (\msf{val} \ v)
    \ e \RANG \longrightarrow \\ \qquad \qquad \LANG w \SEP
    \msf{bindMS} \ \emptyset \ (x, \ v) \ e \RANG
\end{array} }
\end{gather}

\begin{gather}
  \frac{ \phi; \rho \vdash \LANG w \SEP e_1 \RANG \longrightarrow
    \LANG w' \SEP e'_1 \RANG }{ \phi; \rho \vdash \LANG w \SEP
    \msf{bindS} \ x \ e_1 \ e_2 \RANG \longrightarrow \LANG w' \SEP
    \msf{bindS} \ x \ e'_1 \ e_2 \RANG }
\end{gather}


\subsection{bindMS}

\begin{gather}
  \frac{}{ \phi; \rho \vdash \LANG w \SEP \msf{bindMS} \ \_ \ \_
    \ (\msf{val} \ v) \RANG \longrightarrow \LANG w \SEP \msf{val} \ v
    \RANG }
\end{gather}

\begin{gather}
  \frac{ \begin{array}{ll} \phi' / \phi; \rho' / \rho \vdash \LANG w
      \SEP e \RANG \longrightarrow \LANG w \SEP e' \RANG \end{array} }
       { \begin{array}{ll} \phi; \rho \vdash \LANG w \SEP \msf{bindMS}
           \ \phi' \ \rho' \ e \RANG \longrightarrow \LANG w \SEP
           \msf{bindMS} \ \phi' \ \rho' \ e') \RANG \end{array} }
\end{gather}



\subsection{ifThenElse}

\begin{gather}
  \frac{}{ \phi; \rho \vdash \LANG w \SEP \msf{ifThenElse}
    \ (\msf{val} \ (\msf{cst} \ \msf{bool} \ \msf{true})) \ e_1 \ e_2
    \RANG \longrightarrow \LANG w \SEP e_1 \RANG }
\end{gather}

\begin{gather}
  \frac{}{ \phi; \rho \vdash \LANG w \SEP \msf{ifThenElse}
    \ (\msf{val} \ (\msf{cst} \ \msf{bool} \ \msf{false})) \ e_1 \ e_2
    \RANG \longrightarrow \LANG w \SEP e_2 \RANG }
\end{gather}


\begin{gather}
  \frac{ \phi; \rho \vdash \LANG w \SEP e_1 \RANG \longrightarrow
    \LANG w' \SEP e'_1 \RANG }{ \phi; \rho \vdash \LANG w \SEP
    \msf{ifThenElse} \ e_1 \ e_2 \ e_3 \RANG \longrightarrow \LANG w'
    \SEP \msf{ifThenElse} \ e'_1 \ e_2 \ e_3 \RANG }
\end{gather}




\subsection{apply}

\begin{gather}
  \frac{ \vdash \msf{values} \ \mbox{\it es} \ \mbox{\it vs}
  }{ \begin{array}{ll} \phi; \rho \vdash \LANG w \SEP \msf{apply}
      \ (\msf{fun} \ \phi \ \Gamma \ e_0 \ e_1 \ x \ 0) \ (\msf{PS}
      \ \mbox{\it es}) \RANG \longrightarrow \\ \qquad \qquad \qquad
      \LANG w \SEP \msf{bindMS} \ \phi \ (\msf{mkValEnv} \ \Gamma
      \ \mbox{\it vs}) \ e_0 \RANG \end{array}}
\end{gather}

\begin{gather}
  \frac{ \vdash \msf{values} \ \mbox{\it es} \ \mbox{\it vs}
  }{ \begin{array}{ll} \phi; \rho \vdash \LANG w \SEP \msf{apply}
      \ (\msf{fun} \ \phi \ \Gamma \ e_0 \ e_1 \ x \ (\msf{S} \ n))
      \ (\msf{PS} \ \mbox{\it es}) \RANG \longrightarrow \\ \quad
      \LANG w \SEP \msf{bindMS} \ ((x, \ \msf{FC} \ \phi \ \Gamma
      \ e_0 \ e_1 \ x \ n) / \phi) \ (\msf{mkValEnv} \ \Gamma
      \ \mbox{\it vs}) \ e_1 \RANG \end{array}}
\end{gather}

\begin{gather}
  \frac{ \phi; \rho \vdash \LANG w \SEP ps \RANG \longrightarrow \LANG
    w' \SEP ps' \RANG }{ \phi; \rho \vdash \LANG w \SEP \msf{apply}
    \ f \ ps \RANG \longrightarrow \LANG w' \SEP \msf{apply} \ f \ ps'
    \RANG }
\end{gather}

\begin{gather}
  \frac{ \phi \vdash \LANG w \SEP \msf{\it qf} \RANG \longrightarrow
    \LANG w' \SEP \msf{\it qf'} \RANG }{ \phi; \rho \vdash \LANG w
    \SEP \msf{apply} \ \msf{\it qf} \ ps \RANG \longrightarrow \LANG
    w' \SEP \msf{apply} \ \msf{\it qf'} \ ps \RANG }
\end{gather}



\subsection{modify}

\begin{gather}
  \frac{ \vdash \mbox{\it xf} \ : \ \msf{XFun} \ T_1 \ T_2 \qquad
    \vdash \mbox{\it xf}.\msf{b\_mod} \ w \ n \ = \ (w',n') } { \phi;
    \rho \vdash \LANG w \SEP \msf{modify} \ \mbox{\it xf} \ (\msf{cst}
    \ T_1 \ n) \RANG \longrightarrow \LANG w' \SEP \msf{Val}
    \ (\msf{cst} \ T_2 \ n') \RANG }
\end{gather}

\begin{gather}
  \frac{ \rho \vdash \LANG w \SEP qv \RANG \longrightarrow \LANG w'
    \SEP qv' \RANG }{ \phi; \rho \vdash \LANG w \SEP \msf{modify} \ qv
    \RANG \longrightarrow \LANG w' \SEP \msf{modify} \ \mbox{\it xf}
    \ qv' \RANG }
\end{gather}





\subsection{parameters}


\begin{gather}
\frac{\begin{array}{lll}
    \vdash \msf{values} \ \mbox{\it es}_1  \ \_ \\
%
    \vdash ps \ = \ \msf{PS} \ (\msf{append} \ \mbox{\it es}_1
    \ (\msf{cons} \ e \ \mbox{\it es}_2)) \\
%
    \vdash ps' \ = \ \msf{PS} \ (\msf{append} \ \mbox{\it es}_1
    \ (\msf{cons} \ e' \ \mbox{\it es}_2)) \\
%
    \phi; \rho
    \vdash \LANG w \SEP
    e \RANG \longrightarrow \LANG w' \SEP e' \RANG
        \end{array}}
%
  {\phi; \rho
    \vdash \LANG w \SEP
    ps \RANG \longrightarrow \LANG w' \SEP ps' \RANG }
\end{gather}



\subsection{prog}

\begin{gather}
  \frac{ \phi; \rho \vdash \LANG w \SEP e \RANG
    \longrightarrow \LANG w' \SEP e' \RANG }{ \phi; \rho \vdash
    \LANG w \SEP \msf{prog} \ e \RANG \longrightarrow \LANG
    w' \SEP \msf{prog} \ e' \RANG }
\end{gather}



\subsection{define}

\begin{gather}
  \frac{}{ \phi \vdash \LANG w \SEP \msf{define} \ \_ \ \_
    \ (\msf{prog} \ (\msf{val} \ v)) \RANG \longrightarrow \LANG w
    \SEP \msf{prog} \ (\msf{val} \ v) \RANG }
\end{gather}

\begin{gather}
  \frac{ (x, \ f) / \phi \vdash \LANG w \SEP \pi \RANG \longrightarrow
    \LANG w' \SEP \pi' \RANG }{ \phi \vdash \LANG w \SEP \msf{define}
    \ x \ f \ \pi \RANG \longrightarrow \LANG w' \SEP \msf{define} \ x
    \ f \ \pi' \RANG }
\end{gather}




\section{Declarative static semantics }




\subsection{Expression lists}

\begin{gather}
\frac{ }
 { \Phi; \Gamma \vdash \msf{nil} \ :: \ \msf{nil} }
\end{gather}


\begin{gather}
\frac{ \Phi; \Gamma \vdash e \ :: \ t \qquad \Phi; \Gamma \vdash
  \mbox{\it es} \ :: \ ts } { \Phi; \Gamma \vdash \msf{cons} \ e
  \ \mbox{\it es} \ :: \ \msf{cons} \ t \ ts }
\end{gather}




\subsection{Parameters}


\begin{gather}
  \frac{\begin{array}{lll}
    \Phi; \Gamma \vdash \mbox{\it es} \ :: \ ts 
    \end{array}
  }{ \Phi; \Gamma \vdash \msf{PS} \ \mbox{\it es} \ :: \ \msf{PT} \ ts
  }
\end{gather}




\subsection{Functions}


\begin{gather}
  \frac{
\begin{array}{lll}  
  \vdash \phi \ :: \ \Phi \qquad \vdash n \ : \ \msf{nat} \\
  \Phi; \Gamma \vdash e_0 \ :: \ t \\
  (x, \ \Gamma \Rightarrow t) / \Phi; \Gamma \vdash e_1 \ :: \ t \\  
\end{array}  
  }{ \vdash \msf{FC} \ \phi \ \Gamma \ e_0 \ e_1 \ x \ n \ :: \ \Gamma
    \Rightarrow t}
\end{gather}




\subsection{Quasi-functions}


\begin{gather}
\frac{ \vdash \Phi \ x \ = \ \msf{Some} \ \tau}{\Phi \vdash \msf{FVar}
  \ x \ :: \ \tau}
\end{gather}

\begin{gather}
\frac{ \vdash f \ :: \ \tau}{ \Phi \vdash \msf{QF} \ f \ :: \ \tau}
\end{gather}



\subsection{Expressions}

\begin{gather}
\frac{ \vdash v \ :: \ t}{\Phi; \Gamma \vdash \msf{val} \ v :: \ t}
\end{gather}

\begin{gather}
  \frac{ \Gamma \vdash qv \ :: \ t \qquad \vdash \mbox{\it tg} \ :
    \ \msf{Tag} } {\Phi; \Gamma \vdash \msf{return} \ \mbox{\it tg}
    \ qv \ :: \ t}
\end{gather}

\begin{gather}
 \frac{\Phi; \Gamma \vdash e_1 \ :: \ t_1 \qquad \Phi; \Gamma \vdash
   e_2 \ :: \ t_2} {\Phi; \Gamma \vdash \msf{bindN} \ e_1 \ e_2 \ ::
   \ t_2 }
\end{gather}

\begin{gather}
 \frac{\Phi; \Gamma \vdash e_1 \ :: \ t_1 \qquad \Phi; (x, \ t_1) /
   \Gamma \vdash e_2 \ :: \ t_2} {\Phi; \Gamma \vdash \msf{bindS} \ x
   \ e_1 \ e_2 \ :: \ t_2 }
\end{gather}

\begin{gather}
  \frac{ \vdash \phi \ :: \ \Phi' \qquad \vdash \rho \ :: \ \Gamma'
    \qquad \Phi' / \Phi; \Gamma' / \Gamma \vdash e \ :: \ t} {\Phi;
    \Gamma \vdash \msf{bindMS} \ \phi \ \rho \ e \ :: \ t }
\end{gather}

\begin{gather}
  \frac{\Phi; \Gamma \vdash e_1 \ :: \ \msf{vtyp} \ \msf{bool} \qquad
    \Phi; \Gamma \vdash e_2 \ :: \ t \qquad \Phi; \Gamma \vdash e_3
    \ :: \ t } {\Phi; \Gamma \vdash \msf{ifThenElse} \ e_1 \ e_2 \ e_3
    \ :: \ t }
\end{gather}

\begin{gather}
\frac{ \Phi \vdash \msf{\it qf} \ :: \ \Gamma' \Rightarrow t \qquad
  \Phi; \Gamma \vdash ps \ :: \ \msf{tc2pt} \ \Gamma'}{\Phi; \Gamma
  \vdash \ \msf{apply} \ \msf{\it qf} \ ps \ :: \ t}
\end{gather}

\begin{gather}
  \frac{ \vdash \mbox{\it xf} \ : \ \msf{XFun} \ T_1 \ T_2 \ \qquad
    \ \Gamma \vdash qv \ :: \ \msf{vtyp} \ T_1 } {\Phi; \Gamma \vdash
    \msf{modify} \ \mbox{\it xf} \ qv \ :: \ \msf{vtyp} \ T_2}
\end{gather}




\subsection{Programs}

\begin{gather}
 \frac{\Phi \vdash e \ :: \ t} {\Phi \vdash
   \msf{prog} \ e \ :: \ t }
\end{gather}


\begin{gather}
 \frac{ \vdash f \ :: \ \tau \qquad (x, \ \tau) / \Phi \vdash \pi \ ::
   \ t} {\Phi \vdash \msf{define} \ x \ f \ \pi \ :: \ t }
\end{gather}


\section{Reflective translation (outline)}

We give an informal specification of the reflective translation to the
Gallina fragment used in the executable model of Pip. We focus on the
most interesting case -- expression translation ($\Theta_e$). \emph{TC}
stands for the typing context information, and \emph{E} for
environment information (moreover including information about the
external monadic functions). We informally denote by \emph{\msf{ext}
  \ x \ e \ TC} as the extension of \emph{TC} by associating the
identifier $x$ with the type of expression $e$, and by \emph{\msf{ext}
  \ x \ e \ E} as the extension of \emph{E} by associating the
identifier $x$ with the value obtained from $e$. We leave unspecified
the translation of types ($\Theta_t$), values ($\Theta_v$), quasi-values
($\Theta_{qv}$) and quasi-functions ($\Theta_{qf})$, which are not
problematic, the translation of parameters ($\Theta_{ps}$), which
essentially maps expression translation on the underlying list, and
the translation of functions ($\Theta_f$), which boils down to the
translation of the function bodies.

The term $\Theta_e \ e \ TC$ is meant to have type which depends on the
type of $e$ and on $TC$. The term $\Theta_e \ e \ TC$ is meant to have
type which depends solely on the type $t$ of $e$ -- indeed, type
$\Theta_t \ t$.


\begin{gather}
\begin{array}{lll}  
\Theta_e \ (\msf{val} \ v, \_, \_) & = \ \msf{ret} \ (\Theta_v \ v) \\
%
\Theta_e \ (\msf{return} \ \_ \ qv) \ \_ \ E & = \ \msf{ret}
\ (\Theta_{qv} \ qv \ E) \\
%
\Theta_e \ (\msf{bindN} \ e_1 \ e_2) \ TC \ E & = \ \msf{bind}
\ (\Theta_e \ e_1 \ TC \ E) \ (\Theta_e \ e_2 \ TC \ E) \\
%
\Theta_e \ (\msf{bindS} \ x \ e_1 \ e_2) \ TC \ E & = \ \msf{bind}
\ (\Theta_e \ e_1 \ TC \ E) \\ & \qquad (\Theta_e \ e_2 \ (\msf{ext} \ x
\ e_1 \ TC) \ (\msf{ext} \ x \ e_1 \ E)) \\
%
\Theta_e \ (\msf{ifThenElse} \\ \quad (\msf{return} \ qv) \ e_1 \ e_2)
\ TC \ E & = \ \msf{if} \ (\Theta_{qv} \ qv \ E) \ \msf{then}
\ \ (\Theta_e \ e_1 \ TC \ E) \\ & \qquad \qquad \qquad \qquad
\msf{else} \ (\Theta_e \ e_2 \ TC \ E) \\
%
\Theta_e \ (\msf{apply} \ \mbox{\it qf} \ ps) \ TC \ E & =
\ \msf{bind} \ (\Theta_{ps} \ ps \ TC \ E) \ (\Theta_{\it{qf}}
\ \mbox{\it qf} \ TC \ E) \\
%
\Theta_e \ (\msf{modify} \ \mbox{\it xf} \ qv) \ TC \ E & = 
\ (\Theta_{\it{xf}} \ \mbox{\it xf} \ TC \ E) \ (\Theta_{qv} \ qv \ E)
\end{array}
\end{gather}


\section{Translation to C (outline)}


We give an informal specification of the translation to C, again
focusing on expressions. Essentially, the translation returns
recursively a C expression paired with a list of C variable
declarations. Typing information is used pervasively, but omitted in
this schematic presentation. DEC identifiers are translated to C
variables. Sequencing is translated to comma expressions ($,_c$),
local variables to comma expressions with assignment ($=_c$),
conditionals to conditional expressions ($?:$), and function calls to
function calls ($\_ \ \_$). Function translation boils down to
translating the function bodies and lifting them to statements. The
top-level function environment is translated by mapping function
translation on the list.


\begin{gather}
\begin{array}{lll}  
\Xi_e \ (\msf{val} \ v) & = \ (\Xi_v \ v, \ \msf{nil}) \\
%
\Xi_e \ (\msf{return} \ \_ \ v) & = \ (\Xi_{qv} \ v, \ \msf{nil}) \\
%
\Xi_e \ (\msf{bindN} \ e_1 \ e_2) & = \ \msf{let} \ (ee_1, dd_1) \ :=
\ (\Xi_e \ e_1) \\ & \qquad \quad (ee_2, dd_2) \ := \ (\Xi_e \ e_2) \\
& \qquad \msf{in} \
((ee_1 \ ,_c \ ee_2), \ \msf{concat} \ dd_1 \ dd_2) \\
%
\Xi_e \ (\msf{bindS} \ x \ e_1 \ e_2) & = \ \msf{let} \ (ee_1, dd_1)
\ := \ (\Xi_e \ e_1) \\ & \qquad \quad (ee_2, dd_2) \ := \ (\Xi_e
\ e_2) \\ & \qquad \qquad i \ := \ \Xi_i \ x \\ & \qquad
\msf{in} \ ((i \ =_c \ ee_1 \ ,_c \ ee_2), \\ & \qquad \qquad \msf{ext}
\ i \ (\Xi_t \ e_1) \ (\msf{concat} \ dd_1 \ dd_2)) \\
%
\Xi_e \ (\msf{ifThenElse} \\ \qquad (\msf{return} \ qv) \ e_1 \ e_2) &
= \ \msf{let} \ (ee_1, dd_1) \ := \ (\Xi_e \ e_1) \\ & \qquad \quad
(ee_2, dd_2) \ := \ (\Xi_e \ e_2) \\ & \qquad \msf{in} \ ((\Xi_{qv}
\ qv) \ ? \ ee_1 \ : \ ee_2, \\ & \qquad \qquad \msf{concat} \ dd_1
\ dd_2) \\
%
\Xi_e \ (\msf{apply} \ \mbox{\it qf} \ ps) & = \ \msf{let} \ (pps, dd)
\ = \ (\Xi_e \ ps) \\ & \qquad \msf{in} \ ((\Xi_{\it{qf}} \ \mbox{\it
  qf}) \ pps, \ dd) \\
%
\Xi_e \ (\msf{modify} \ \mbox{\it xf} \ qv) & = \ ((\Xi_{\it{xf}}
\ \mbox{\it xf}) \ (\Xi_{qv} \ qv), \ \msf{nil})
\end{array}
\end{gather}



We can refine the above translation function to one which targets
CompCert C (\texttt{http://compcert.inria.fr/compcert-C.html}). We can
then rely on the CompCert representation of C expressions as a
datatype (in \emph{Csyntax.expr}). Here the translation returns
recursively a C expression paired with its type, which is needed by
CompCert, and a list of C global declarations. The latter results in a
list of CompCert type $(\msf{ident} \ * \ \msf{globdef} \ \msf{fundef}
\ \msf{type})$, given the translation of DEC identifiers to C
identifiers. As before, sequencing is translated to comma expressions
(\msf{Ecomma}), local variables to comma expressions with assignment
(\msf{Eassign}), conditionals to conditional expressions
(\msf{Econdition}), and function calls to function calls
(\msf{Ecall}).


\begin{gather}
\begin{array}{lll}  
\Psi_e \ (\msf{val} \ v) & = \ (\Psi_v \ v, \ \msf{nil}) \\
%
\Psi_e \ (\msf{return} \ \_ \ v) & = \ (\Psi_{qv} \ v, \ \msf{nil}) \\
%
\Psi_e \ (\msf{bindN} \ e_1 \ e_2) & = \ \msf{let} \ (ee_1, \ \_,
\ dd_1) \ := \ (\Psi_e \ e_1) \\ & \qquad \quad (ee_2, \ ty, \ dd_2)
\ := \ (\Psi_e \ e_2) \\ & \qquad \msf{in} \ ((\msf{Ecomma} \ ee_1
\ ee_2 \ ty), \\ & \qquad \qquad ty, \ \msf{concat} \ dd_1 \ dd_2) \\
%
\Psi_e \ (\msf{bindS} \ x \ e_1 \ e_2) & = \ \msf{let} \ (ee_1, \ ty_1,
\ dd_1) \ := \ (\Psi_e \ e_1) \\ & \qquad \quad (ee_2, \ ty_2, \ dd_2)
\ := \ (\Psi_e \ e_2) \\ & \qquad \quad i \ := \ \Psi_i \ x \\ & \qquad
\msf{in} \ (\msf{Ecomma} \ (\msf{Eassign} \ i \ ee_1 \ ty_1) \ ee_2
\ ty_2, \\ & \qquad \qquad ty_2, \ \msf{cons} \ (i,\ ty_1)
\ (\msf{concat} \ dd_1 \ dd_2)) \\
%
\Psi_e \ (\msf{ifThenElse} \\ \qquad  (\msf{return} \ qv) \
 e_1 \ e_2) & =
\ \msf{let} \ (vv, \ \msf{Tint} \ \msf{IBool} \ \_ \ \_) \ :=
\ (\Psi_{qv} \ qv) \\ & \qquad \quad (ee_1, ty, \ dd_1) \ := \ (\Psi_e
\ e_1) \\ & \qquad \quad (ee_2, \ ty, \ dd_2) \ := \ (\Psi_e \ e_2)
\\ & \qquad \msf{in} \ (\msf{Econdition} \ vv \ ee_1 \ ee_2 \ ty,
 \\ & \qquad \qquad ty, \ \msf{concat} \ dd_1 \ dd_2) \\
%
\Psi_e \ (\msf{apply} \ \mbox{\it qf} \ ps) & = \ \msf{let} \ (pps,
\ tts, \ dd) \ := \ (\Psi_{ps} \ ps) \\ & \qquad \quad (\it{ff},
\ \msf{Tfunction} \ tys \ ty \ \_) \ := \ (\Psi_{fs} \ qf) \\ & \qquad
\msf{in} \ (\msf{Ecall} \ \it{ff} \ pps \ ty, \ ty, \ dd) \\
%
\Psi_e \ (\msf{modify} \ \mbox{\it xf} \ qv) & = \ \msf{let} \ tys
\ := \ \msf{Tcons} \ tt \ \msf{Tnil} \\ & \qquad \quad (qqv, \ tts)
\ := \ (\Psi_{qv} \ qv) \\ & \qquad \quad (\it{ff}, \ \msf{Tfunction}
\ tys \ ty \ \_) \ := \ (\Psi_{\it{xf}} \ \mbox{\it xf}) \\ & \qquad
\msf{in} \ (\msf{Ecall} \ \it{ff} \ qv \ ty, \ ty, \ \msf{nil})
\end{array}
\end{gather}



\end{document}

